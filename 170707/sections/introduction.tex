\section{Introduction}
\label{intro}
	Machine to machine (M2M) communication, which is also denoted as machine type communication (MTC) in 3GPP, is an emerging concept that allow devices to communicate with MTC servers or each other automatically without or with minimal human interaction~\cite{3GPP22.368}, and the goal of the concept is to provide comprehensive connections among all smart devices. By using the long term evolution (LTE) network, M2M devices can exchange their data through the network more quickly and efficiently, and it not only short the distance between we humans but also make our life more convenient. Furthermore, it enables faster and more reliable communications than traditional human initiated communications~\cite{lien2011toward}.

	
    With the rapid development of M2M or MTC, more and more applications such as smart grid, real-time health services, etc are growing vigorously in the past few years~\cite{3GPP22.868}~\cite{3GPP23.888} and it can be predicted that the number of M2M devices will expand to billion in 2020~\cite{ABI}. Based on their design objective, MTC applications can be broadly classified into several categories: metering, telehealth, building security, asset tracking. Different applications such as the real-time health services which need strict access delay and high access success probability and smart grid which perform periodic access and do not need too strict delay (moderate delay) are geared toward improving the delay, access success probability, or energy efficiency. How to efficiently managing access for different MTC devices is also a hot issue that we concern about~\cite{rajandekar2015survey}.
    

    As suggested in~\cite{3GPP37.868}~\cite{hasan2013random}~\cite{cao2013cellular}~\cite{gotsis2012m2m}, one of the most critical challenges is the network access congestion problem. The deployment of massive MTC devices may cause congestion problem because of the exploded signal flow on radio access networks (RANs) and core networks (CNs) while the very large number of MTC devices try to access the network simultaneously. In consequently, the effect of the problem would result in terrible delay, waste of resources and low access success rate, so overload control mechanisms are necessary for random-based MTC devices to improve the performance.
	

    In order to alleviate the RAN overload, we focus on the objective that can increase access success probability and relieve the access delays. A new grouping idea of ACB scheme is proposed to solve the problem. However, the traditional ACB scheme proposed in~\cite{R2-103742} has one of weakness that the ACB factor (Access probability) is fixed and the eNB cannot change the ACB factor according to different RAN overload level in time. Therefore, we proposed a heuristic algorithm that eNB can estimate the number of MTC devices who perform the preamble transmissions through the number of preamble transmissions send by MTC devices who successfully perform random access so that it is able to change the ACB factor dynamically.


    The rest of this work is organized as follows. The related work and background is described in section~\ref{Background and Related Work}. In section~\ref{system model}, we present the system model. Section~\ref{proposed} describes the idea that how we let each MTC device know which groups it belongs in every random access occasion. Section~\ref{Mathmatics} describes the method that eNB dynamically resets the barring factor. In section~\ref{simulation}, simulation results are presented to compare our proposed scheme with the traditional scheme. The work is concluded in section~\ref{conclusion}.

    