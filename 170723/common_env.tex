


% 除非校方修改了論文格式 (margins, header, footer, 浮水印)
% 或者需要增加所用的 LaTeX 套件,
% 或者要改預設中文字型、編碼
% 否則毋須修改本檔內容
% 論文撰寫,請修改以 my_  開頭檔名的各檔案




\usepackage[nospace]{cite}  % for smart citation
\usepackage{geometry}  % for easy margin settings
%\usepackage{subfigure}  % for subfigure  因為有衝突
%\usepackage[dvipdfm]{graphicx}  % for graphic   using eps
%\usepackage[xetex]{graphicx}
%\usepackage{graphicx}  % for graphic   using eps

%\usepackage{epstopdf} % 當使用pdflatex時打開,如使用latex則不需開啟,此功能為將xxx.eps 自動判讀為XXX.pdf


%\usepackage{algorithmic}  %演算法使用
\usepackage{algorithm}
\usepackage{algorithmicx}
\usepackage{algpseudocode}
\usepackage{amsmath} % 各式 AMS 數學功能
\usepackage{mathrsfs} %草寫體數學符號,在數學模式裡用 \mathscr{E} 得草寫 E
\usepackage{listings} % 程式列表套件
\usepackage{url} % 在文稿中引用網址,可以用 \url{http://www.ntust.edu.tw} 方式
\usepackage{fancyhdr}  % 借用增強功能型 header 套件來擺放浮水印 
