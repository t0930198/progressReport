\documentclass{beamer}
\usetheme{ConnectivityLab}
\usepackage{times}
\usepackage{graphicx}
\usepackage{verbatim}
\usepackage{outlines}
\usepackage{fancyhdr}
\usepackage{subfigure}
\usepackage{cancel}
\usepackage{bibentry}
\usepackage{varwidth}
\usepackage{etoolbox}
\usepackage{epstopdf}

%%%%%%%%%%%%%%%%%%%%%%%%%%%%%%%%%%%%%%%%%%%%%%%%%%%%%%
%%%%%%%%%%%%%%%%%%%%%%%%%%%%%%%%%%%%%%%%%%%%%%%%%%%%%%

\title {
    A Probably Secure Ring Signature Scheme in Certificateless Cryptography \cite{2017arXiv171209145Z}
}
\author {
    Yin-Hong Hsu
}
\date {
    01 10, 2018
}

%%%%%%%%%%%%%%%%%%%%%%%%%%%%%%%%%%%%%%%%%%%%%%%%%%%%%%
%%%%%%%%%%%%%%%%%%%%%%%%%%%%%%%%%%%%%%%%%%%%%%%%%%%%%%

\begin{document}
\begin{frame}
    \titlepage
\end{frame}

%%%%%%%%%%%%%%%%%%%%%%%%%%%%%%%%%%%%%%%%%%%%%%%%%%%%%%
%%%%%%%%%%%%%%%%%%%%%%%%%%%%%%%%%%%%%%%%%%%%%%%%%%%%%%

\begin{frame}{Outline}
    \tableofcontentsgather
    \tableofcontents
\end{frame}

%%%%%%%%%%%%%%%%%%%%%%%%%%%%%%%%%%%%%%%%%%%%%%%%%%%%%%
%%%%%%%%%%%%%%%%%%%%%%%%%%%%%%%%%%%%%%%%%%%%%%%%%%%%%%
\section{Preliminaries}
\begin{frame} {Preliminaries} 
    \begin{itemize}
        \item {certificateless ring signature}
        \begin{itemize}
                \item [-]{setup}
                \item [-]{partial-private-key-extract}
                \item [-]{set-secret-value}
                \item [-]{set-private-key}
                \item [-]{set-public-key}
                \item [-]{ring-sign}
                \item [-]{verify}
        \end{itemize}
    \end{itemize}
\end{frame}
\section{Security Model}
\begin{frame}{Security Model}
    \begin{itemize}
        \item {Type I Adverary}
        \begin{itemize}
                \item [-]{replace user's public key}
        \end{itemize}
        \item {Type II Adverary}
        \begin{itemize}
                \item [-]{access to the master key (which be used to be a part of private key)}
        \end{itemize}
    \end{itemize}
\end{frame}
\begin{frame}{Security Model}
    \begin{itemize}
        \item {Game I: Unforgeability of CL-Ring against Type I Adversery}
        \item {Game II: Unforgeability of CL-Ring against Type II Adversery}
    \end{itemize}
\end{frame}
\section{Analysis of A Generic Construction of CL-Ring}
\begin{frame}{Analysis of A Generic Construction of CL-Ring}
    \begin{itemize}
        \item {Construction of CL-Ring}
        \begin{itemize}
                \item [-]{Detail of 7 functions were mentioned in page 2}
        \end{itemize}
        \item {And some security analysis with Game I and II}
    \end{itemize}
\end{frame}
\section{A Concrete CL-Ring Scheme}
\begin{frame}{A Concrete CL-Ring Scheme}
    \begin{itemize}
        \item {Different algorithm for these 7 functions}
        \item {The number of pairing computation is constant and dose not grow with the number of group members}
    \end{itemize}
\end{frame}
%%%%%%%%%%%%%%%%%%%%%%%%%%%%%%%%%%%%%%%%%%%%%%%%%%%%%%
%%%%%%%%%%%%%%%%%%%%%%%%%%%%%%%%%%%%%%%%%%%%%%%%%%%%%%
\section{References}
\calcreferencespagetotal % Calc your References Page total number
\begin{frame}[allowframebreaks]{References}
    \fontsize{9pt}{13}\selectfont
    \bibliographystyle{IEEEtran}
    \bibliography{IEEEabrv,Citation}
\end{frame}

%%%%%%%%%%%%%%%%%%%%%%%%%%%%%%%%%%%%%%%%%%%%%%%%%%%%%%
%%%%%%%%%%%%%%%%%%%%%%%%%%%%%%%%%%%%%%%%%%%%%%%%%%%%%%
\section{}

\begin{frame}
    \centering
    \Large{Thanks for Your Attentions}
\end{frame}

\end{document}
